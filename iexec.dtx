% \iffalse meta-comment
% (The MIT License)
%
% Copyright (c) 2021-2022 Yegor Bugayenko
%
% Permission is hereby granted, free of charge, to any person obtaining a copy
% of this software and associated documentation files (the 'Software'), to deal
% in the Software without restriction, including without limitation the rights
% to use, copy, modify, merge, publish, distribute, sublicense, and/or sell
% copies of the Software, and to permit persons to whom the Software is
% furnished to do so, subject to the following conditions:
%
% The above copyright notice and this permission notice shall be included in all
% copies or substantial portions of the Software.
%
% THE SOFTWARE IS PROVIDED 'AS IS', WITHOUT WARRANTY OF ANY KIND, EXPRESS OR
% IMPLIED, INCLUDING BUT NOT LIMITED TO THE WARRANTIES OF MERCHANTABILITY,
% FITNESS FOR A PARTICULAR PURPOSE AND NONINFRINGEMENT. IN NO EVENT SHALL THE
% AUTHORS OR COPYRIGHT HOLDERS BE LIABLE FOR ANY CLAIM, DAMAGES OR OTHER
% LIABILITY, WHETHER IN AN ACTION OF CONTRACT, TORT OR OTHERWISE, ARISING FROM,
% OUT OF OR IN CONNECTION WITH THE SOFTWARE OR THE USE OR OTHER DEALINGS IN THE
% SOFTWARE.
% \fi

% \CheckSum{0}
%
% \CharacterTable
%  {Upper-case    \A\B\C\D\E\F\G\H\I\J\K\L\M\N\O\P\Q\R\S\T\U\V\W\X\Y\Z
%   Lower-case    \a\b\c\d\e\f\g\h\i\j\k\l\m\n\o\p\q\r\s\t\u\v\w\x\y\z
%   Digits        \0\1\2\3\4\5\6\7\8\9
%   Exclamation   \!     Double quote  \"     Hash (number) \#
%   Dollar        \$     Percent       \%     Ampersand     \&
%   Acute accent  \'     Left paren    \(     Right paren   \)
%   Asterisk      \*     Plus          \+     Comma         \,
%   Minus         \-     Point         \.     Solidus       \/
%   Colon         \:     Semicolon     \;     Less than     \<
%   Equals        \=     Greater than  \>     Question mark \?
%   Commercial at \@     Left bracket  \[     Backslash     \\
%   Right bracket \]     Circumflex    \^     Underscore    \_
%   Grave accent  \`     Left brace    \{     Vertical bar  \|
%   Right brace   \}     Tilde         \~}

% \GetFileInfo{iexec.dtx}
% \DoNotIndex{\endgroup,\begingroup,\let,\else,\fi,\newcommand,\newenvironment}

% \iffalse
%<*driver>
\ProvidesFile{iexec.dtx}
%</driver>
%<package>\NeedsTeXFormat{LaTeX2e}
%<package>\ProvidesPackage{iexec}
%<*package>
[0000-00-00 0.0.0 Inputable Shell Executions]
%</package>
%<*driver>
\documentclass{ltxdoc}
\usepackage[tt=false, type1=true]{libertine}
\usepackage{microtype}
\usepackage[dtx]{docshots}
\usepackage{iexec}
\usepackage{href-ul}
\PageIndex
\EnableCrossrefs
\CodelineIndex
\RecordChanges
\begin{document}
	\DocInput{iexec.dtx}
	\PrintChanges
	\PrintIndex
\end{document}
%</driver>
% \fi

% \title{|iexec|: \LaTeX{} Package \\ for Inputable Shell Executions\thanks{The sources are in GitHub at \href{https://github.com/yegor256/iexec}{yegor256/iexec}}}
% \author{Yegor Bugayenko \\ \texttt{yegor256@gmail.com}}
% \date{\filedate, \fileversion}
%
% \maketitle
%
% \textbf{NB!}
% This package doesn't work on Windows!
%
% \section{Introduction}
%
% This package helps you execute shell commands right from the
% document and then put their output to the document:
%
% \begin{docshot}
% \documentclass{article}
% \usepackage{iexec}
% \usepackage[paperwidth=3in]{geometry}
% \pagestyle{empty}
% \begin{document}
% Today is \textbf{%
%   \iexec{date +\%e-\%b-\%Y}}
% \end{document}
% \end{docshot}

% \DescribeMacro{\iexec} The only command provided by this package is
% |\iexec| \oarg{options} \marg{cmd}.
% Its only mandatory argument \meta{cmd} is the command to be executed
% through the terminal shell (|bash|, or whatever is set as the default one
% in your console).

% You have to run |pdflatex| (or just |latex|)
% with the |--shell-escape| flag
% in order to let \href{https://ctan.org/pkg/shellesc}{shellesc}
% execute your shell command.

% \section{Options}

% \DescribeMacro{quiet}
% If you don't want the output to be visible,
% use |\phantom\{\iexec{...}}|.
% Otherwise, you can use |quiet| option:
%\iffalse
%<*verb>
%\fi
\begin{verbatim}
I just want to delete some file:
\iexec[quiet]{rm -f foo.txt}
\end{verbatim}
%\iffalse
%</verb>
%\fi
% In this case, whatever the shell command produces will not be included
% into the document.

% \DescribeMacro{stdout}
% The output of your code is saved into the file provided as an
% optional argument of |\iexec| (the default value is |iexec.tmp|):
%\iffalse
%<*verb>
%\fi
\begin{verbatim}
Today is \iexec[stdout=date.txt]{date +\%e-\%b-\%Y | tr -d '\\n'}.
\end{verbatim}
%\iffalse
%</verb>
%\fi
% The tailing part of the command here removes all ends-of-line.

% \DescribeMacro{stderr}
% The error output of the code is saved into the file provided as an
% optional argument of |\iexec| (by default the error output
% is streamed into |stdout|):
%\iffalse
%<*verb>
%\fi
\begin{verbatim}
Today is \iexec[stderr=my.txt]{broken-command}.
\end{verbatim}
%\iffalse
%</verb>
%\fi

% \DescribeMacro{exit}
% The exit code of the command it saved into a file. You can change the name of it
% using |exit| option:
%\iffalse
%<*verb>
%\fi
\begin{verbatim}
Today is \iexec[exit=code.txt]{./broken-command.sh}.
\end{verbatim}
%\iffalse
%</verb>
%\fi

% \DescribeMacro{trace}
% The file specified will be deleted right after its usage. If you don't
% want this to happen, use |trace| package option: all files will remain
% in the directory where they were created.
% It's possible to turn tracing on globbaly, for the entire document, using
% |trace| option of the package:
%\iffalse
%<*verb>
%\fi
\begin{verbatim}
\documentclass{article}
\usepackage[trace]{iexec}
\begin{document}
This file won't be deleted: \iexec[stdout=me.txt]{whoami}.
\end{document}
\end{verbatim}
%\iffalse
%</verb>
%\fi

% \DescribeMacro{append}
% The stdout produced will be appended to the file specified:
%\iffalse
%<*verb>
%\fi
\begin{verbatim}
\documentclass{article}
\usepackage[trace]{iexec}
\begin{document}
\iexec[append,stdout=foo.txt,quiet]{echo 'Hello, '}
\iexec[append,stdout=foo.txt,quiet]{echo 'Jeffrey!'}
\input{foo.txt}
\end{document}
\end{verbatim}
%\iffalse
%</verb>
%\fi

% \DescribeMacro{log}
% The stdout produced will be printed in \TeX{} log:
%\iffalse
%<*verb>
%\fi
\begin{verbatim}
\iexec[log]{echo 'Hello, \\LaTeX!'}
\end{verbatim}
%\iffalse
%</verb>
%\fi

% \DescribeMacro{null}
% The stdout of the command will be sent to |/dev/null|:
%\iffalse
%<*verb>
%\fi
\begin{verbatim}
\iexec[null]{rm some-file.txt}
\end{verbatim}
%\iffalse
%</verb>
%\fi

% \DescribeMacro{ignore}
% By default, we report an error if exit code is not equal to zero. You can suppress
% this with |ignore| option:
%\iffalse
%<*verb>
%\fi
\begin{verbatim}
\iexec[ignore]{broken-command}
\end{verbatim}
%\iffalse
%</verb>
%\fi

% \StopEventually{}

% \section{Implementation}

% First, we include \href{https://ctan.org/pkg/shellesc}{shellesc} package, which
% we use to execute shell commands:
%    \begin{macrocode}
\RequirePackage{shellesc}
%    \end{macrocode}

% Then, we parse package options:
%    \begin{macrocode}
\RequirePackage{xkeyval}
\makeatletter
\newif\ifiexec@trace
\DeclareOptionX{trace}{\iexec@tracetrue}
\ProcessOptionsX\relax
\makeatother
%    \end{macrocode}

% Then, we prepare to parse the options of |\iexec| command:
% \changes{0.10.0}{2022/10/19}{The option "ignore" suppresses the checking of "iexec.ret" value.}
% \changes{0.7.0}{2022/09/25}{The option "append" was introduced --- if it's turned on, stdout will be appended to the file, instead of rewriting it (this is how it was before).}
% \changes{0.7.0}{2022/09/25}{The option "log" was introduced, to turn on log/debug messages in TeX log (they were all visible always, which was sometimes annoying. Also, this option enables printing of the entire content of stdout to the log too (this may be pretty convenient for debugging).}
% \changes{0.11.0}{2022/10/22}{The option "exit" allows to change the name of the file with exit code.}
%    \begin{macrocode}
\RequirePackage{pgfkeys}
\makeatletter\pgfkeys{
  /iexec/.is family,
  /iexec,
  exit/.store in = \iexec@exit,
  exit/.default = iexec.ret,
  stdout/.store in = \iexec@stdout,
  stdout/.default = iexec.tmp,
  stderr/.store in = \iexec@stderr,
  trace/.store in = \iexec@traceit,
  append/.store in = \iexec@append,
  log/.store in = \iexec@log,
  null/.store in = \iexec@null,
  quiet/.store in = \iexec@quiet,
  ignore/.store in = \iexec@ignore,
  stdout,exit
}\makeatother
%    \end{macrocode}

% \begin{macro}{\iexec@typeout}
% Then, we define an internal command |\iexec@typeout| for printing the content of a file,
% as suggested \href{https://tex.stackexchange.com/questions/660808}{here}:
%    \begin{macrocode}
\RequirePackage{expl3}
\makeatletter\ExplSyntaxOn
\NewDocumentCommand{\iexec@typeout}{m}{
  \iexec_typeout_file:n { #1 }}
\ior_new:N \g_iexec_typeout_ior
\cs_new_protected:Nn \iexec_typeout_file:n
{
  \ior_open:Nn \g_iexec_typeout_ior { #1 }
  \ior_str_map_inline:Nn \g_iexec_typeout_ior
    {\iow_term:n { ##1 }}
  \ior_close:N \g_iexec_typeout_ior
}
\ExplSyntaxOff\makeatother
%    \end{macrocode}
% \end{macro}

% \begin{macro}{\iexec}
% Then, we define |\iexec| command.
% It is implemented with the help of |\ShellEscape| from |shellesc| package:
% \changes{0.10.0}{2022/10/19}{The file "iexec.ret" is reused for all scripts.}
%    \begin{macrocode}
\makeatletter
\newread\iexec@exitfile
\newcommand\iexec[2][]{%
  \begingroup%
    \pgfqkeys{/iexec}{#1}%
%    \end{macrocode}
% First, we verify that |latex| is running with |--shell-escape| option, since without
% it nothing will work; so, it's better to throw an error earlier than later:
%    \begin{macrocode}
    \ifnum\ShellEscapeStatus=1\else%
      \PackageError{iexec}{You must run TeX processor with
      --shell-escape option}{}%
    \fi%
    \begingroup%
%    \end{macrocode}
% Then, start the log from a clean line:
%    \begin{macrocode}
      \ifdefined\iexec@log%
        \message{^^J}%
      \fi%
%    \end{macrocode}
% Then, we define a few special chars in order to escape them in the shell
% (the full
% list of them is in \href{https://ctan.mirror.norbert-ruehl.de/info/macros2e/macros2e.pdf}{macros2e}):
%    \begin{macrocode}
      \let\%\@percentchar%
      \let\\\@backslashchar%
      \let\{\@charlb%
      \let\}\@charrb%
%    \end{macrocode}
% Then, we execute it:
% \changes{0.10.0}{2022/10/19}{The ability to track exit code was added. Now, the code is saved into "iexec.ret" file, which is then read and checked for zero value.}
% \changes{0.8.0}{2022/10/05}{The option "null" was introduced, allowing redirection of stdout to "/dev/null". Doesn't work on Windows, though.}
% \changes{0.9.0}{2022/10/15}{The option "stderr" was introduced, allowing redirection of stderr to a file. Without this option specified, stderr will go to stdout.}
% \changes{0.11.0}{2022/10/22}{The file with exit code now contains just numbers, without end of line.}
%    \begin{macrocode}
      \def\iexec@cmd{(#2)
        \ifdefined\iexec@append>\fi>
        \ifdefined\iexec@null/dev/null\else\iexec@stdout\fi
        \space\ifdefined\iexec@stderr2>\iexec@stderr\else2>&1\fi;
        /bin/echo -n $?\% >\iexec@exit}
      \ShellEscape{\iexec@cmd}%
%    \end{macrocode}
% Then, a message is printed to TeX log:
%    \begin{macrocode}
      \ifdefined\iexec@log%
        \message{iexec: [\iexec@cmd]^^J}%
      \fi%
    \endgroup%
%    \end{macrocode}
% Then, if required, the content of the stdout file will be printed to the log:
%    \begin{macrocode}
    \ifdefined\iexec@null\else%
    \ifdefined\iexec@log%
      \message{iexec: This is the content of \iexec@stdout:^^J}%
      \iexec@typeout{\iexec@stdout}%
      \message{<EOF>^^J}%
    \fi\fi%
%    \end{macrocode}
% Then, we check exit code, unless there is |ignore| option:
%    \begin{macrocode}
    \immediate\openin\iexec@exitfile=\iexec@exit%
    \read\iexec@exitfile to \iexec@code%
    \immediate\closein\iexec@exitfile%
    \ifnum\iexec@code=0\else%
      \ifdefined\iexec@ignore%
        \ifdefined\iexec@log%
          \message{iexec: Execution failure ignored,
            the exit code was \iexec@code^^J}%
        \fi%
      \else%
        \PackageError{iexec}{Execution failure,
          the exit code was \iexec@code}{}%
      \fi%
    \fi%
%    \end{macrocode}
% Then, include the produced output into the current document:
%    \begin{macrocode}
    \ifdefined\iexec@null\else%
    \ifdefined\iexec@quiet%
      \ifdefined\iexec@log%
        \message{iexec: Due to 'quiet' option we didn't read
        the content of '\iexec@stdout'
        \ifdefined\pdffilesize (\pdffilesize{\iexec@stdout}
        bytes)\fi^^J}%
      \fi%
    \else%
      \ifdefined\iexec@log%
        \message{iexec: We are going to include the content of
        '\iexec@stdout'\ifdefined\pdffilesize (\pdffilesize
        {\iexec@stdout} bytes)\fi...^^J}%
      \fi%
      \input{\iexec@stdout}%
      \message{iexec: The content of '\iexec@stdout'
      was included into the document^^J}%
    \fi\fi%
%    \end{macrocode}
% Finally, delete the file or leave it untouched:
%    \begin{macrocode}
    \ifdefined\iexec@null\else%
    \ifiexec@trace%
      \ifdefined\iexec@log%
        \message{iexec: Due to package option 'trace',
        the files '\iexec@stdout' and `\iexec@exit` were
        not deleted^^J}%
      \fi%
    \else%
      \ifdefined\iexec@traceit%
        \ifdefined\iexec@log%
          \message{iexec: Due to 'trace' package option,
          the files '\iexec@stdout' and '\iexec@exit'
          were not deleted^^J}%
        \fi%
      \else%
        \ShellEscape{rm \iexec@stdout}%
        \ifdefined\iexec@log%
          \message{iexec: The file '\iexec@stdout' was deleted^^J}%
        \fi%
        \ShellEscape{rm \iexec@exit}%
        \ifdefined\iexec@log%
          \message{iexec: The file '\iexec@exit' was deleted^^J}%
        \fi%
      \fi%
    \fi\fi%
  \endgroup
}\makeatother
%    \end{macrocode}
% \end{macro}

% \Finale

%\clearpage
%\PrintChanges

%\clearpage
%\PrintIndex
